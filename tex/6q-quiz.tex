\documentclass[11pt]{article}
\usepackage[cm]{fullpage}
%%AVC PACKAGES
\usepackage{avcgreek}
\usepackage{avcfonts}
\usepackage{avcmath}
\usepackage{avcthm}
\usepackage{qcmacros}
\usepackage{goldstone}
%%MACROS FOR THIS DOCUMENT
\usepackage[
  margin=1.5cm,
  includefoot,
  footskip=30pt,
  headsep=0.2cm,headheight=1.3cm
]{geometry}
\usepackage{fancyhdr}
\pagestyle{fancy}
\fancyhf{}
\fancyhead[LE,RO]{\textbf{Quiz 6}}
\fancyfoot[CE,CO]{\thepage}
\usepackage{url}

\begin{document}

\begin{enumerate}
\item
Derive the recursive equation for the wavefunction, starting from the $\la$-dependent Schr\"odinger equation.
\begin{align}
\label{eq:recursive-wavefunction-equation}
  \Y(\la)
=
  \F
+
  R_0
  (
    \la V_\mr{c}
  -
    E(\la)
  )
  \Y(\la)
\end{align}
Assume intermediate normalization and note that $R_0H_0=-Q$ follows** from the definition of $R_0$.\\[10pt]
**\textbf{Extra Credit}: Define ``resolvent'' and explain why this follows from your definition.


\newpage
\item
Determine the first- and second-order components of $\Y$ by differentiating equation~\ref{eq:recursive-wavefunction-equation}.
You do \ul{not} need to fully evaluate and simplify your answer,\footnote{That is, your final answer may contain $R_0$'s and $V_\mr{c}$'s.} but you should eliminate all terms that vanish and explain why each one evaluates to zero.\footnote{You may take $E_\mr{c}\ord{m+1}=\ip{\F|V_\mr{c}|\Y\ord{m}}$ as given.}


\newpage
\item
Evaluate the following contributions to the CI doubles and quadruples coefficients.
\begin{align}
  {}\ord{1}
  c_{ab}^{ij}
=
  \ip{\F_{ij}^{ab}|
    R_0V_\mr{c}
  |\F}
&&
  {}\ord{2}
  c_{abcd}^{ijkl}
=
  \ip{\F_{ijkl}^{abcd}|
    R_0V_\mr{c}R_0V_\mr{c}
  |\F}
\end{align}
Use your answer to show that ${}\ord{2}C_4=\tfr{1}{2}{}\ord{1}C_2^2$.


\end{enumerate}

\end{document}
