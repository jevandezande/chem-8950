\documentclass[11pt]{article}
\usepackage[cm]{fullpage}
%%AVC PACKAGES
\usepackage{avcgreek}
\usepackage{avcfonts}
\usepackage{avcmath}
\usepackage[numberby=section,skip=9pt plus 2pt minus 7pt]{avcthm}
\usepackage{qcmacros}
\usepackage{goldstone}
%%MACROS FOR THIS DOCUMENT
\numberwithin{equation}{section}
\usepackage[
  margin=1.5cm,
  includefoot,
  footskip=30pt,
  headsep=0.2cm,headheight=1.3cm
]{geometry}
\usepackage{fancyhdr}
\pagestyle{fancy}
\fancyhf{}
\fancyhead[LE,RO]{Quiz 7, Handout 1: Perturbative analysis}
\fancyfoot[CE,CO]{\thepage}
\usepackage{url}
\makeatother
\newcommand{\resolventline}[2][1]{
  \tikz[overlay]{
      \draw[thick,flexdotted] (0,-1ex) to ++(0,#1*4.5ex) node[above,inner sep=1pt] {#2};
  }
}

\newcommand{\fwkt}[1]{\kt{\makebox[0.8em][c]{\ensuremath{#1}}}}
\newcommand{\fwbr}[1]{\br{\makebox[0.8em][c]{\ensuremath{#1}}}}
\newcommand{\bord}[1]{\ensuremath{^{[#1]}}}

\begin{document}

\setlength{\abovedisplayskip}{3pt}
\setlength{\belowdisplayskip}{3pt}



\setcounter{section}{6}
\section{Perturbative analysis}


\begin{dfn}
\thmtitle{Correct to order $p$}
Define $X\ord{p^-}\equiv X\ord{0}+X\ord{1}+\cd+X\ord{p}$
as well as
$
  X\ord{p^+}
\equiv
  X\ord{p}
+
  X\ord{p+1}
+
  \cd
+
  X\ord{\infty}
$.
Then we say an approximation to $X$ is \textit{correct to order $p$} if it contains all of the contributions in $X\ord{p^-}$.
\end{dfn}

\begin{dfn}
\thmtitle{Truncated CI and CC}
Let $\mr{CISDTQPH78}{\cd}m$ denote truncation of the CI Ansatz at $m$-tuples.
Similarly, $\mr{CCS}{\cd}m$ means that we omit cluster operators of excitation level $m\leq k\leq n$.
Note that, unlike $C_k$, the cluster operator $T_k$ in truncated CC describes only \textit{connected $k$-tuples}, which are residual correlations that cannot be decomposed into products of smaller clusters.
As a result, truncated CC actually contains much higher excitations than CI.
\end{dfn}

\begin{ntt}
Let
$\bm{\F}_k$
be a row vector containing all unique $k$-fold substitutions of $\F$,\footnote{That is, $\bm{\F}_k$ contains all unique $\F_{i_1\cd i_k}^{a_1\cd a_k}$.
Uniqueness can be enforced by requiring $i_1<\cd<i_k$ and $a_1<\cd<a_k$.}
so that $\bm{\F}=\pma{\F\ \bm{\F}_1\,\cd\,\bm{\F}_n}$ spans $\mc{F}_n$.
Using $\bo{c}$ and $\bo{t}$ to denote column vectors of CI coefficients and CC amplitudes leads to the following relationships.\footnote{For the sake of generality we are not assuming intermediate normalization for CI.  The $C_0$ operator scales functions by $c_0$.}
\begin{align}
  \bo{\F}\cdot\bo{c}
=
  (C_0 + C_1+\cd +C_n)\F
&&
  \bo{\F}\cdot\bo{t}
=
  (1 + T_1+\cd +T_n)\F
\end{align}
In Dirac notation the bra $\br{\bo{\F}}$ is transposed, so that
$
  \ip{\bm{\F}|W|\bm{\F}}
=
  [\ip{\F_\si|W|\F_\ta}]
$
is the matrix representation of $W$ in $\mc{F}_n$.
Such matrix representations will be denoted with bolded letters,
$
  \bo{W}
\equiv
  \ip{\bm{\F}|W|\bm{\F}}
$.
\end{ntt}



\begin{rmk}
\label{rmk:ci-perturbative-analysis}
\thmtitle{Perturbative analysis of CI}
Writing the CI eigenvalue equation $\bo{H}_\mr{c}\,\bo{c}=E_\mr{c}\,\bo{c}$ in terms of model-Hamiltonian and fluctuation-potential matrices and rearranging yields a new matrix equation
\begin{align}
\label{eq:ci-matrix-equation}
  (
  -
    \bo{H}_0
  +
    E_\mr{c}
  )\,
  \bo{c}
=
  \bo{V}_\mr{c}\,
  \bo{c}
\end{align}
which provides a convenient starting point for a perturbative analysis and for comparison to the coupled-cluster equations.
The matrix elements of the model Hamiltonian are given by
$
  \ip{\F_\si|H_0|\F_\ta}
=
  \mc{E}_\si
  \d_{\si\ta}
$,
so the matrix on the left is diagonal with eigenvalues
$
-
  \mc{E}_{i_1\cd i_k}^{a_1\cd a_k}
+
  E_\mr{c}
$.
The rows of this equation can be written in terms of CI operators as follows\footnote{See \url{https://en.wikipedia.org/wiki/Floor_and_ceiling_functions} for details floor and ceiling functions, $\floor{x}$ and $\ceil{x}$.}
\begin{align}
\label{eq:ci-reference-equation}
  \underset{(0^+)}{\vphantom{(}
  c_0
  }
  \underset{(2^+)}{\vphantom{(}
    E_\mr{c}
  }
=&\
  \ip{\F|
  \underset{(1)}{\vphantom{(}
    V_\mr{c}
  }
    (
    \underset{((1^+))}{\vphantom{(}
      C_1
    }
    +
    \underset{(1^+)}{\vphantom{(}
      C_2
    }
    )
  |\F}
\\
\label{eq:ci-singles-equation}
  \underset{(1^+)}{\vphantom{(}
  c_a^i
  }
  (\hspace{1pt}
  \underset{(0)}{\vphantom{(}
    \mc{E}_a^i
  }
  +
  \underset{(2^+)}{\vphantom{(}
    E_\mr{c}
  }
  \hspace{-2pt}
  )
=&\
  \ip{\F_i^a|
  \underset{(1)}{\vphantom{(}
    V_\mr{c}
  }
    (
    \underset{((0^+))}{\vphantom{(}
      C_0
    }
    +
    \underset{(1^+)}{\vphantom{(}
      C_1
    }
    +
    \underset{(1^+)}{\vphantom{(}
      C_2
    }
    +
    \underset{(2^+)}{\vphantom{(}
      C_3
    }
    )
  |\F}
\\
\label{eq:ci-doubles-equation}
  \underset{(1^+)}{\vphantom{(}
  c_{ab}^{ij}
  }
  (\hspace{1pt}
  \underset{(0)}{\vphantom{(}
    \mc{E}_{ab}^{ij}
  }
  +
  \underset{(2^+)}{\vphantom{(}
    E_\mr{c}
  }
  \hspace{-2pt}
  )
=&\
  \ip{\F_{ij}^{ab}|
  \underset{(1)}{\vphantom{(}
    V_\mr{c}
  }
    (
    \underset{(0^+)}{\vphantom{(}
      C_0
    }
    +
    \underset{(1^+)}{\vphantom{(}
      C_1
    }
    +
    \underset{(1^+)}{\vphantom{(}
      C_2
    }
    +
    \underset{(2^+)}{\vphantom{(}
      C_3
    }
    +
    \underset{(2^+)}{\vphantom{(}
      C_4
    }
    )
  |\F}
\\
  \underset{(2^+)}{\vphantom{(}
  c_{abc}^{ijk}
  }
  (\hspace{1pt}
  \underset{(0)}{\vphantom{(}
    \mc{E}_{abc}^{ijk}
  }
  +
  \underset{(2^+)}{\vphantom{(}
    E_\mr{c}
  }
  \hspace{-2pt}
  )
=&\
  \ip{\F_{ijk}^{abc}|
  \underset{(1)}{\vphantom{(}
    V_\mr{c}
  }
    (
    \underset{(1^+)}{\vphantom{(}
      C_1
    }
    +
    \underset{(1^+)}{\vphantom{(}
      C_2
    }
    +
    \underset{(2^+)}{\vphantom{(}
      C_3
    }
    +
    \underset{(2^+)}{\vphantom{(}
      C_4
    }
    +
    \underset{(3^+)}{\vphantom{(}
      C_5
    }
    )
  |\F}
\\
  \underset{(2^+)}{\vphantom{(}
  c_{abcd}^{ijkl}
  }
  (\hspace{1pt}
  \underset{(0)}{\vphantom{(}
    \mc{E}_{abcd}^{ijkl}
  }
  +
  \underset{(2^+)}{\vphantom{(}
    E_\mr{c}
  }
  \hspace{-2pt}
  )
\underset{\hspace{2pt}\vdots}{\vphantom{(}
=
}&\
  \ip{\F_{ijkl}^{abcd}|
  \underset{(1)}{\vphantom{(}
    V_\mr{c}
  }
    (
    \underset{(1^+)}{\vphantom{(}
      C_2
    }
    +
    \underset{(2^+)}{\vphantom{(}
      C_3
    }
    +
    \underset{(2^+)}{\vphantom{(}
      C_4
    }
    +
    \underset{(3^+)}{\vphantom{(}
      C_5
    }
    +
    \underset{(3^+)}{\vphantom{(}
      C_6
    }
    )
  |\F}
\\
\label{eq:ci-k-tuples-equation}
  \underset{(\ceil{k/2}^+)}{\vphantom{(}
  c_{a_1\cd a_k}^{i_1\cd i_k}
  }
  (\hspace{1pt}
  \underset{(0)}{\vphantom{(}
    \mc{E}_{a_1\cd a_k}^{i_1\cd i_k}
  }
  +
  \underset{(2^+)}{\vphantom{(}
    E_\mr{c}
  }
  \hspace{-2pt}
  )
=&\
  \ip{\F_{i_1\cd i_k}^{a_1\cd a_k}|
  \underset{(1)}{\vphantom{(}
    V_\mr{c}
  }
    (
    \underset{(\ceil{k/2}^+-1)}{\vphantom{(}
      C_{k-2}
    }
    +
    \underset{(\ceil{(k-1)/2}^+)}{\vphantom{(}
      C_{k-1}
    }
    +
    \underset{(\ceil{k/2}^+)}{\vphantom{(}
      C_k
    }
    +
    \underset{(\ceil{(k+1)/2}^+)}{\vphantom{(}
      C_{k+1}
    }
    +
    \underset{(\ceil{k/2}^++\,1)}{\vphantom{(}
      C_{k+2}
    }
    )
  |\F}
\end{align}
where the numbers in parentheses denote orders in perturbation theory and the double parentheses denote terms which vanish under Brillouin's theorem.
The orders of the CI operators follow from the fact that each order in perturbation theory increases the maximum excitation level of the wavefunction by $+2$, starting from $\Y\ord{1}$ which contains up to doubles.
Therefore the leading contributions to $C_k$ have order $\ceil{k/2}$.
If Brillouin's theorem holds, the first-order contribution to $C_1$ vanishes and singles contribute at orders $2^+$ in perturbation theory.
\end{rmk}


\begin{ex}
\label{ex:analysis-of-cisd-throuh-q}
The analysis of remark~\ref{rmk:ci-perturbative-analysis} shows that CISD is only correct to first order in the wavefunction, since triples contribute at second order.
However, $C_1$ and $C_2$ are both correct to second order, since the truncation error in equations \ref{eq:ci-singles-equation} and \ref{eq:ci-doubles-equation} is $\mc{O}(V_\mr{c}^3)$, making the CISD correlation energy correct to third order.
In order to gain an order in perturbation theory we have to increase the truncation level by at least two, since triples and quadruples contribute at the same order.
CISDTQ is correct to second order in the wavefunction and fifth order in the energy.
\end{ex}

\begin{prop}
\label{prop:ci-orders}
\thmstatement{
$\mr{CIS}{\cd}m$
is correct to order
$\floor{m/2}$
in the wavefunction and order
$2\floor{m/2}+1$
in the energy.  
}
\thmproof{
According to \cref{rmk:ci-perturbative-analysis}, $C_{m+1}$ contributes at order $\ceil{(m+1)/2}$,
implying that the wavefunction is correct to
$
  \ceil{(m+1)/2}
-
  1
=
  \floor{m/2}
$.
Truncation also leaves $C_m$ and $C_{m-1}$ correct to
$
  \ceil{(m+1)/2}
=
  \floor{m/2}
+
  1
$,
and propagating these truncation errors down to
$C_{m-2h}$
and
$C_{m-1-2h}$
makes the latter correct to
$
  \floor{m/2}
+
  1
+
  h
$.
One of these operators is $C_2$ when
$
  h
=
  \floor{m/2}
-
  1
$.
Since $C_2$ limits the error in equation~\ref{eq:ci-reference-equation},\,\footnote{The error propagation ensures that errors decrease monontonically with excitation level.} the energy is correct to
$
  2\floor{m/2}
+
  1
$.
}
\end{prop}

\begin{rmk}
\label{rmk:cc-perturbative-analysis}
\thmtitle{Perturbative analysis of CC}
The CC equations can be written as a non-linear matrix equation similar.
\begin{align}
\label{eq:cc-matrix-equation}
  E_\mr{c}
  \ip{\bm{\F}|\F}
-
  \bo{H}_0\hspace{0.8pt}
  \bo{t}
=
  \ip{\bm{\F}|
    V_\mr{c}\,
    \mr{exp}(T(\bo{t}))
  |\F}_\mr{C}
\end{align}
To make the comparison with equation~\ref{eq:ci-matrix-equation} more transparent, this can be written as
$
(
-
  \bo{H}_0
+
  E_\mr{c}\,
  \ip{\bm{\F}|\F}
)
  \bo{t}
=
  (
    \bo{V}_\mr{c}\hspace{0.8pt}
    \bo{t}
  )_{\mr{C}}
+
  \mc{O}(\bo{t}^2)
$.
Non-vanishing contributions to the rows of this matrix equation can be expressed in terms of cluster operators as follows
\begin{align}
\label{eq:cc-reference-equation}
  \underset{(2^+)}{\vphantom{(}
  E_\mr{c}
  }
=&\
  \ip{\F|
  \underset{(1)}{\vphantom{(}
    V_\mr{c}
  }
    (
    \underset{((1^{+}))}{\vphantom{(}
      T_1
    }
    +
    \underset{(1^{+})}{\vphantom{(}
      T_2
    }
    +
      \tfr{1}{2}
    \underset{(2^{+})}{\vphantom{(}
      T_1^2
    }
    )
  |\F}_\mr{C}
\\[3pt]
  \underset{(1^+)}{\vphantom{(}
  t_a^i
  }
  \underset{(0)}{\vphantom{(}
  \mc{E}_a^i
  }
=&\
  \ip{\F_i^a|
  \underset{(1)}{\vphantom{(}
    V_\mr{c}
  }
    (
    \underset{((0))}{\vphantom{(}
      1
    }
    +
    \underset{(1^{+})}{\vphantom{(}
      T_1
    }
    +
    \underset{(1^{+})}{\vphantom{(}
      T_2
    }
    +
    \underset{(2^{+})}{\vphantom{(}
      T_3
    }
    +
      \tfr{1}{2}
    \underset{(2^{+})}{\vphantom{(}
      T_1^2
    }
    +
    \underset{(2^{+})}{\vphantom{(}
      T_1T_2
    }
    +
      \tfr{1}{3!}
    \underset{(3^{+})}{\vphantom{(}
      T_1^3
    }
    )
  |\F}_\mr{C}
\\[3pt]
\nonumber
  \underset{(1^+)}{\vphantom{(}
  t_{ab}^{ij}
  }
  \underset{(0)}{\vphantom{(}
  \mc{E}_{ab}^{ij}
  }
=&\
  \br{\F_{ij}^{ab}}
  \underset{(1)}{\vphantom{(}
    V_\mr{c}
  }
    (
    \underset{(0)}{\vphantom{(}
      1
    }
    +
    \underset{(1^{+})}{\vphantom{(}
      T_1
    }
    +
    \underset{(1^{+})}{\vphantom{(}
      T_2
    }
    +
    \underset{(2^{+})}{\vphantom{(}
      T_3
    }
    +
    \underset{(3^{+})}{\vphantom{(}
      T_4
    }
    +
      \tfr{1}{2}
    \underset{(2^{+})}{\vphantom{(}
      T_1^2
    }
    +
    \underset{(2^{+})}{\vphantom{(}
      T_1T_2
    }
    +
    \underset{(3^{+})}{\vphantom{(}
      T_1T_3
    }
\\
\label{eq:cc-doubles-equation}
&
\makebox[0.75\linewidth][r]{\ensuremath{
    +
      \tfr{1}{2}
    \underset{(2^{+})}{\vphantom{(}
      T_2^2
    }
    +
      \tfr{1}{3!}
    \underset{(3^{+})}{\vphantom{(}
      T_1^3
    }
    +
      \tfr{1}{2}
    \underset{(3^{+})}{\vphantom{(}
      T_1^2T_2
    }
    +
      \tfr{1}{4!}
    \underset{(4^{+})}{\vphantom{(}
      T_1^4
    }
    )
  \kt{\F}_\mr{C}
}}
\\[3pt]
\nonumber
  \underset{(2^+)}{\vphantom{(}
  t_{abc}^{ijk}
  }
  \underset{(0)}{\vphantom{(}
  \mc{E}_{abc}^{ijk}
  }
=&\
  \br{\F_{ijk}^{abc}}
  \underset{(1)}{\vphantom{(}
    V_\mr{c}
  }
    (
    \underset{(1^{+})}{\vphantom{(}
      T_2
    }
    +
    \underset{(2^{+})}{\vphantom{(}
      T_3
    }
    +
    \underset{(3^{+})}{\vphantom{(}
      T_4
    }
    +
    \underset{(4^{+})}{\vphantom{(}
      T_5
    }
    +
    \underset{(2^{+})}{\vphantom{(}
      T_1T_2
    }
    +
    \underset{(3^{+})}{\vphantom{(}
      T_1T_3
    }
    +
      \tfr{1}{2}
    \underset{(2^{+})}{\vphantom{(}
      T_2^2
    }
    +
    \underset{(4^{+})}{\vphantom{(}
      T_1T_4
    }
    +
    \underset{(3^{+})}{\vphantom{(}
      T_2T_3
    }
\\
\label{eq:cc-triples-equation}
&
\makebox[0.75\linewidth][r]{\ensuremath{
    +
      \tfr{1}{2}
    \underset{(3^{+})}{\vphantom{(}
      T_1^2T_2
    }
    +
      \tfr{1}{2}
    \underset{(4^{+})}{\vphantom{(}
      T_1^2T_3
    }
    +
      \tfr{1}{2}
    \underset{(3^{+})}{\vphantom{(}
      T_1T_2^2
    }
    +
      \tfr{1}{3!}
    \underset{(4^{+})}{\vphantom{(}
      T_1^3T_2
    }
    )
  \kt{\F}_\mr{C}
}}
\\[3pt]
\nonumber
  \underset{(3^+)}{\vphantom{(}
  t_{abcd}^{ijkl}
  }
  \underset{(0)}{\vphantom{(}
  \mc{E}_{abcd}^{ijkl}
  }
=&\
  \br{\F_{ijkl}^{abcd}}
  \underset{(1)}{\vphantom{(}
    V_\mr{c}
  }
    (
    \underset{(2^{+})}{\vphantom{(}
      T_3
    }
    +
    \underset{(3^{+})}{\vphantom{(}
      T_4
    }
    +
    \underset{(4^{+})}{\vphantom{(}
      T_5
    }
    +
    \underset{(5^{+})}{\vphantom{(}
      T_6
    }
    +
    \underset{(2^{+})}{\vphantom{(}
      T_1T_3
    }
    +
      \tfr{1}{2}
    \underset{(2^{+})}{\vphantom{(}
      T_2^2
    }
    +
    \underset{(4^{+})}{\vphantom{(}
      T_1T_4
    }
    +
    \underset{(3^{+})}{\vphantom{(}
      T_2T_3
    }
    +
    \underset{(5^{+})}{\vphantom{(}
      T_1T_5
    }
    +
    \underset{(4^{+})}{\vphantom{(}
      T_2T_4
    }
    +
      \tfr{1}{2}
    \underset{(4^{+})}{\vphantom{(}
      T_3^2
    }
\\
\underset{\displaystyle\vdots}{{}}\hspace{2pt}
&
\makebox[0.75\linewidth][r]{\ensuremath{
    +
      \tfr{1}{2}
    \underset{(4^{+})}{\vphantom{(}
      T_1^2T_3
    }
    +
      \tfr{1}{2}
    \underset{(3^{+})}{\vphantom{(}
      T_1T_2^2
    }
    +
      \tfr{1}{2}
    \underset{(5^{+})}{\vphantom{(}
      T_1^2T_4
    }
    +
    \underset{(4^{+})}{\vphantom{(}
      T_1T_2T_3
    }
    +
      \tfr{1}{3!}
    \underset{(3^{+})}{\vphantom{(}
      T_2^3
    }
    +
      \tfr{1}{3!}
    \underset{(5^{+})}{\vphantom{(}
      T_1^3T_3
    }
    +
      \tfr{1}{2!2!}
    \underset{(4^{+})}{\vphantom{(}
      T_1^2T_2^2
    }
    )
  \kt{\F}_\mr{C}
}}
\\[3pt]
  \underset{((k-1)^+)}{\vphantom{(}
  t_{a_1\cd a_k}^{i_1\cd i_k}
  }
  \underset{(0)}{\vphantom{(}
  \mc{E}_{a_1\cd a_k}^{i_1\cd i_k}
  }
=&\
  \br{\F_{i_1\cd i_k}^{a_1\cd a_k}}
  \underset{(1)}{\vphantom{(}
    V_\mr{c}
  }
    (
    \underset{((k-2)^{+})}{\vphantom{(}
      T_{k-1}
    }
    +
    \underset{((k-1)^{+})}{\vphantom{(}
      T_k
    }
    +
    \hspace{2pt}
    \underset{(k^{+})}{\vphantom{(}
      T_{k+1}
    }
    \hspace{1pt}
    +
    \underset{((k+1)^{+})}{\vphantom{(}
      T_{k+2}
    }
    +
      \sum_{p=2}^4
      \fr{1}{p!}
      \sum_{h=p-2}^{2}
      \sum_{\bm{k}}^{\mc{C}_p(k+h)}
    \underset{((k+h-p+\f_{\bm{k}})^+)}{\vphantom{(}
      T_{k_1}
      \cd
      T_{k_p}
    }
    )
  \kt{\F}_\mr{C}
\end{align}
where $\mc{C}_k(m)$ is the set of $k$-part compositions of $m$
and $\f_{\bm{k}}$ is the number of 1's in
$
  \bm{k}
=
  (k_1,\ld,k_p)
$.
The orders of the cluster operators follow from straightforward induction on the fact that the lowest order contribution to $T_k$ always comes from
$
(
  V_{\mr{c}}
  T_{k-1}
)_\mr{C}
$,
which means that each $T_k$ contributes at one order above $T_{k-1}$, starting from $k=2$.
If Brillouin's theorem holds, $T_1$ contributes at orders $2^+$ and the orders of the disconnected products become $(k+h-p+2\cdot\f_{\bm{k}})^+$.
\end{rmk}

\begin{prop}
\label{prop:cc-orders}
\thmstatement{
$\mr{CCS}{\cd}m$ is correct to order $m-1$ in the wavefunction and order $m+\floor{m/2}$ in the energy.
}
\thmproof{
  According to \cref{rmk:cc-perturbative-analysis}, $T_{m+1}$ contributes at order $m$, implying that the wavefunction is correct to order $m-1$.
  Truncation also leaves $T_m$ and $T_{m-1}$ correct to order $m$, and propagating these truncation errors down to $T_{m-2h}$ and $T_{m-1-2h}$ makes the latter correct to $m+h$.
  One of these operators is $T_2$ when $h=\floor{m/2}-1$.
  Since $T_2$ limits the error in equation \ref{eq:cc-reference-equation}, the energy is correct to order $m+\floor{m/2}$.
}
\end{prop}

\begin{ex}
Props~\ref{prop:ci-orders} and \ref{prop:cc-orders} allow us to compare the accuracies of CI and CC in perturbation theory.
Truncating at doubles, CI and CC are both correct to first order in the wavefunction and third order in the energy.
Triples yield no improvement for CI, whereas CC gains an order in both wavefunction and energy.
In general, the CC wavefunction and energy improves upon the CI wavefunction and energy by
$
  m
-
  \floor{m/2}
-
  1
=
  \floor{(m-1)/2}
$
orders in perturbation theory.
\end{ex}

\begin{dfn}
\thmtitle{Order $p$ truncation}
If $X$ is a polynomial in $T_1,T_2,\ld,T_n$,
its \textit{order $p$ truncation}, denoted $X\bord{p}$, retains all terms in the polynomial with leading contributions of order $p$ or less.
This makes $X\bord{p}$ correct to order $p$ without isolating specific orders in the cluster operators, which will generally involve contributions up to infinite order.
\end{dfn}


\begin{samepage}
\begin{ex}
\thmtitle{The $[\mr{T}]$ correction}
Assuming Brillouin's theorem, we can complete the energy to fourth order using
\begin{align}
  t_{ab}^{ij}
\,{=}\,
  \br{\F_{ij}^{ab}}
    R_0
    V_\mr{c}
    (
      1
    \,{+}\,
      T_1
    \,{+}\,
      T_2
    \,{+}\,
      T_3^{[2]}
    \,{+}\,
      \tfr{1}{2}
      T_1^2
    \,{+}\,
      T_1T_2
    \,{+}\,
      \tfr{1}{2}
      T_2^2
    \,{+}\,
      \tfr{1}{3!}
      T_1^3
    \,{+}\,
      \tfr{1}{2}
      T_1^2T_2
    \,{+}\,
      \tfr{1}{4!}
      T_1^4
    )
  \kt{\F}_\mr{C}
&&
  {}\bord{2}
  t_{abc}^{ijk}
\,{=}\,
  \br{\F_{ijk}^{abc}}
    R_0
    V_\mr{c}
    T_2
  \kt{\F}_\mr{C}
\end{align}
where the resulting energy correction is
$
  E_e
-
  E_e^{\mr{CCSD}}
=
  \ip{\F|
    V_\mr{c}
    R_0
    V_\mr{c}
    T_3\bord{2}
  |\F}
$.
We can introduce additional infinite order contributions by noting that
$
  T_2
  \F
=
  R_0
  V_\mr{c}
  \F
+
  \mc{O}(V_\mr{c}^2)
$
and that the additional terms in
$
  \ip{\F|
    T_2\dg
    V_\mr{c}
    T_3\bord{2}
  |\F}
$
are also valid energy contributions in perturbation theory.
There is no risk of double counting since all of these contributions involve connected triples, which are absent in CCSD.
With converged CCD or CCSD $T_2$-amplitudes, this defines the \textit{``brackets'' $\mr{T}$ correction}
\begin{align}
  E_{[\mr{T}]}
=
\diagram{
  \interaction{3}{t*}{(0,+0.5)}{ddot}{overhang};
  \interaction{3}{t}{(0,-0.5)}{ddot}{overhang};
  \draw[->-=0.4,bend left ] (t1) to (t*1);
  \draw[-<-=0.6,bend right] (t1) to (t*1);
  \draw[->-=0.4,bend left ] (t2) to (t*2);
  \draw[-<-=0.6,bend right] (t2) to (t*2);
  \draw[->-=0.4,bend left ] (t3) to (t*3);
  \draw[-<-=0.6,bend right] (t3) to (t*3);
  \draw[thick,dash dot dot] (-0.3,0) to ++(2.6,0);
  \node[right=3pt of t3 ] {[2]};
  \node[right=3pt of t*3] {[2]};
}
=
  (
  \tfr{1}{3!}
  )^2
  \sum_{\substack{abc\\ijk}}
  {}\bord{2}
  t_{abc}^{ijk}{}^*
  \mc{E}_{abc}^{ijk}
  {}\bord{2}
  t_{abc}^{ijk}
&&
\diagram{
  \interaction{3}{t}{(0,-0.5)}{ddot}{overhang};
  \draw[->-] (t1) to ++(-0.25,1);
  \draw[-<-] (t1) to ++(+0.25,1);
  \draw[->-] (t2) to ++(-0.25,1);
  \draw[-<-] (t2) to ++(+0.25,1);
  \draw[->-] (t3) to ++(-0.25,1);
  \draw[-<-] (t3) to ++(+0.25,1);
  \node[right=2pt of t3 ] {[2]};
}\hspace{-10pt}
\equiv
\diagram{
  \interaction{2}{t}{(0,-0.5)}{ddot}{overhang};
  \draw[->-] (t1) to ++(-0.25,1);
  \draw[-<-] (t1) to ++(+0.25,1);
  \draw[->-] (t2) to ++(-0.25,1);
  \draw[-<-=0.25,-<-=0.85]
      (t2)
    to
      node[ddot,midway] (g1) {}
    ++(+0.25,1);
  \draw[sawtooth] (g1) to ++(1,0) node[ddot] (g2) {};
  \draw[->-=0.65] (g2) to ++(-0.25,0.5);
  \draw[-<-=0.65] (g2) to ++(+0.25,0.5);
  \draw[thick,flexdotted] (-0.3,0.25) to ++(2.7,0);
}
+
\diagram{
  \interaction{2}{t}{(0,-0.5)}{ddot}{overhang};
  \draw[-<-] (t1) to ++(-0.25,1);
  \draw[->-] (t1) to ++(+0.25,1);
  \draw[-<-] (t2) to ++(-0.25,1);
  \draw[->-=0.25,->-=0.85]
      (t2)
    to
      node[ddot,midway] (g1) {}
    ++(+0.25,1);
  \draw[sawtooth] (g1) to ++(1,0) node[ddot] (g2) {};
  \draw[-<-=0.65] (g2) to ++(-0.25,0.5);
  \draw[->-=0.65] (g2) to ++(+0.25,0.5);
  \draw[thick,flexdotted] (-0.3,0.25) to ++(2.7,0);
}
\end{align}
which is
$
  \ip{\F|
    T_2\dg
    V_\mr{c}
    T_3\bord{2}
  |\F}
=
-
  \ip{\F|
    T_3\bord{2}{}\dg
    H_0
    T_3\bord{2}
  |\F}
$.
The dash-dotted line represents $-H_0$ and is analogous to a resolvent line.
\end{ex}
\end{samepage}


\begin{rmk}
\thmtitle{CC excited states via equation-of-motion theory}
Similarity transformations preserve operator eigenvalues, so $\ol{H}_e$ has the same spectrum as the original Hamiltonian, $H_e$.
The corresponding eigenvalue equations
\begin{align}
\label{eq:eom-matrix-equations}
  \ol{\bo{H}}_e
  \bo{r}_k
=
  E_k
  \bo{r}_k
&&
  \bo{l}_k\dg
  \ol{\bo{H}}_e
=
  \bo{l}_k\dg
  E_k
&&
  \bo{l}_k^*\cdot
  \bo{r}_l
=
  \d_{kl}
\end{align}
involve both right- and left-eigenvectors since 
$
  \ol{\bo{H}}_e
=
  \ip{\bo{\F}|(H_e\,\mr{exp}(T))_\mr{C}|\bo{\F}}
$
is non-Hermitian.
These can also be written as
\begin{align}
  \ol{H}_e
  \mc{R}_k
  \kt{\F}
=
  E_k
  \mc{R}_k
  \kt{\F}
&&
  \br{\F}
  \ol{H}_e
  \mc{L}_k
=
  \br{\F}
  \mc{L}_k
  E_k
&&
\begin{array}{r@{\ }l}
  \mc{R}_k
  \kt{\F}
&=
  \kt{\bo{\F}}\cdot\bo{r}_k
\\
  \br{\F}
  \mc{L}_k
&=
  \bo{l}_k^*\cdot \br{\bo{\F}}
\end{array}
\end{align}
in terms of abstract states.
$E_k$ denotes the $k\eth$ excitation energy, which matches the corresponding CI eigenvalue unless we truncate the Ansatz.
In general, the expectation value of an observable can be determined as
$
  \ip{\Y_k|W|\Y_k}
=
  \ip{\F|\mc{L}_k\ol{W}\mc{R}_k|\F}
$.
\end{rmk}

\begin{dfn}
\thmtitle{The CC Lagrangian}
Assuming we have solved equation~\ref{eq:cc-matrix-equation}, the right eigenvector of the ground state is simply a unit vector
$
  \bo{r}_e
=
  \ip{\bo{\F}|\F}
$.
The left ground-state eigenvector is unknown, but the biorthonormality condition in equation~\ref{eq:eom-matrix-equations} implies that its first entry is one.
Therefore, the left and right ground-state wave operators have the form
\begin{align}
  \mc{R}_e
=
  1
&&
  \mc{L}_e
=
  1
+
  \La
&&
  \La
=
  \La_1
+
  \cd
+
  \La_n
&&
  \La_k
\equiv
  (\tfr{1}{k!})^2\,
  \la_{i_1\cd i_k}^{a_1\cd a_k}
  \tl{a}^{i_1\cd i_k}_{a_1\cd a_k}
\end{align}
where $\La$ is a linear de-excitation operator analogous to $C\dg$.
The ground-state expectation value is therefore
\begin{align}
  E_e
=
  \ip{\Y_e|H_e|\Y_e}
=
  \ip{\F|
    (
      1
    +
      \La
    )
    \ol{H}_e
  |\F}
\end{align}
which can be identified as a Lagrangian for the coupled-cluster energy.
To see why, note that setting the $\bm{\la}$-gradient equal to zero yields the CC amplitude equations: $\ip{\bo{\F}|\ol{H}_e|\F}=0$.
If these are satisfied, then
$
  E_e
=
  \ip{\F|\ol{H}_e|\F}
$
gives the CC energy.
Therefore, we can view the $\bm{\la}$ as Lagrange multipliers enforcing equation~\ref{eq:cc-matrix-equation} and reduce ground-state CC to a constrained optimization problem.
\end{dfn}

\begin{dfn}
\thmtitle{The CC lambda equations}
Projecting the effective Schr\"odinger equation for the left ground-state eigenbra\footnote{
That is, 
$
  \br{\F}
  (
    1
  +
    \La
  )
  \ol{H}_e
=
  \br{\F}
  (
    1
  +
    \La
  )
  E_e
$.
}
on the right by $\kt{\F_{i_1\cd i_k}^{a_1\cd a_k}}$
and rearranging yields the following form of the \textit{coupled-cluster lambda equations}
\begin{align}
  \la_{i_1\cd i_k}^{a_1\cd a_k}
  \mc{E}_{a_1\cd a_k}^{i_1\cd i_k}
=
  \br{\F}
  \La
  H_0
  T
+
  V_\mr{c}
+
  V_\mr{c}\,
  T
+
  \La
  V_\mr{c}\,
  \mr{exp}(T)
  \kt{\F_{i_1\cd i_k}^{a_1\cd a_k}}_\mr{C}
\end{align}
where we have used $\la_{i_1\cd i_k}^{a_1\cd a_k}E_\mr{c}=\ip{\F|\La(V_\mr{c}\,\mr{exp}(T))_\mr{C}|\F_{i_1\cd i_k}^{a_1\cd a_k}}_\mr{U}$ to cancel the unlinked contributions and the subscript $C$ denotes that both the operators in $\La$ and those in $\mr{exp}(T)$ must be connected to $V_\mr{c}$.
\end{dfn}

\begin{ex}
Assuming Brillouin's theorem holds true, the CCD lambda equations have the following form.
\begin{align}
  \la_{ij}^{ab}
  \mc{E}_{ab}^{ij}
=
  \ip{\F|
    V_\mr{c}
  +
    \La_2
    V_\mr{c}
  +
    \La_2 V_\mr{c}T_2
  |\F_{ij}^{ab}}_\mr{C}
\end{align}
\end{ex}




\newpage

\begin{dfn}
\thmtitle{The $[\mr{T}]$ correction}
For simplicity, let $T$ be the CCSD cluster operator and let
$\ol{H}_\mr{c}$
and
$E_\mr{c}$
be the corresponding effective Hamiltonian and correlation energy.
Then the \textit{``brackets'' $\mr{T}$ correction} can be derived by introducing a unitary exponential triples operator $e^{T_3-T_3\dg}$ in front of the CCSD wave operator,
$
  \Y_{\mr{CCSD[T]}}
=
  e^T
  e^{T_3 - T_3\dg}
  \F
$,
and determining its lowest-order contribution to the energy.
Multiplying the Schr\"odinger equation by the inverse wave operator,
projecting by $\br{\F}$,\,\footnote{
These two steps yield
$
  E_\mr{c}
+
  E_{[\mr{T}]}
+
  \mc{O}(V_\mr{c}^5)
=
  \ip{\F|
  e^{T_3\dg - T_3}\,
  \ol{H}_\mr{c}
  e^{T_3 - T_3\dg}
  |\F}
=
  E_\mr{c}
+
  \ip{\F|T_3\dg \ol{H}_\mr{c}|\F}
+
  \ip{\F|T_3\dg{}^2 \ol{H}_c|\F}
+
  \ip{\F|T_3\dg \ol{H}_cT_3|\F}
+
  \mc{O}(T_3^3)
$.
}
subtracting off the original CCSD energy,
and keeping all terms up to fifth order in perturbation theory yields
\begin{align}
  E_{[\mr{T}]}
=
  \ip{\F|
    T_3\dg
    (
      V_\mr{c}
      T_2
    )_\mr{C}
  |\F}
\end{align}
which corrects the energy 
\end{dfn}


\begin{dfn}
Let $T$, $\ol{H}_\mr{c}$, and $E_\mr{c}$ be the cluster operator, effective Hamiltonian, and correlation energy for a truncated coupled-cluster Ansatz, $\mr{CCSD}{\cd}m$, with $m$ greater than two.
Then the
\textit{``brackets'' $m+1$ correction}, denoted $E_{[m+1]}$, can be derived by introducing a unitary exponential to the wave operator,
$
$.

\begin{align}
=
  \ip{\F|
  e^{T_{m+1}\dg - T_{m+1}}\,
  (H_\mr{c}\mr{exp}(T))_\mr{C}\,
  e^{T_{m+1} - T_{m+1}\dg}
  |\F}
\approx
  E_\mr{c}
+
  \ip{\F|
  T_{m+1}\dg
  (V_\mr{c}\,T_2)_\mr{C}
  |\F}
+
  \mc{O}
\end{align}

\end{dfn}


\begin{rmk}
From PT, we know that the lowest order contributions to $T_1$, $T_2$, and $T_3$ (or $C_1$ and $C_2$) occur at first and second order in perturbation theory.

\end{rmk}

Now, just to be confusing, redefine some shit.
\begin{align}
  \bo{1}_\mr{i}
\equiv
\pma[l]{
  \bo{1} & \bo{0} \\
  \bo{0} & \bo{0}
}
&&
  \bo{1}_\mr{e}
\equiv
\pma[l]{
  \bo{0} & \bo{0} \\
  \bo{0} & \bo{1}
}
&&
  \bo{H}_\mr{xy}
\equiv
  \bo{1}_\mr{x}\,
  \bo{H}\,
  \bo{1}_\mr{y}
&&
  \bo{c}_\mr{x}
\equiv
  \bo{1}_\mr{x}\,
  \bo{c}
\end{align}

\begin{align}
  \bo{H}
=
  \bo{H}_\mr{ii}
+
  \bo{H}_\mr{ie}
+
  \bo{H}_\mr{ei}
+
  \bo{H}_\mr{ee}
\end{align}

\begin{align}
  \bo{R}_\mr{ee}
\equiv
\left.
  (
    E
  -
    \bo{H}
  )^{-1}
\right|_\mr{e}
&&
  \bo{R}_\mr{ee}
  (
    E
  -
    \bo{H}
  )
  \bo{1}_\mr{e}
=
  \bo{1}_\mr{e}
&&
\begin{array}{r@{\ }l}
  \bo{R}_\mr{ee}\,
  (
    E
  -
    \bo{H}
  )
&=
-
  \bo{R}_\mr{ee}\,
  \bo{H}_\mr{ei}
+
  \bo{1}_\mr{e}
\\
  (
    E
  -
    \bo{H}
  )\,
  \bo{R}_\mr{ee}
&=
-
  \bo{H}_\mr{ie}\,
  \bo{R}_\mr{ee}
+
  \bo{1}_\mr{e}
\end{array}
\end{align}
Operating the upper equation on $\bo{c}$ gives zero due to the Schr\"odinger equation, which implies
$
  \bo{c}_\mr{e}
=
  \bo{R}_\mr{ee}
  \bo{H}_\mr{ei}
  \bo{c}_\mr{i}
$.
Projecting the Schr\"odinger equation by $\bo{1}_\mr{i}$ and substituting in this result leads to the following.
\begin{align}
  (
    \bo{H}_\mr{ii}
  +
    \bo{V}_\mr{ii}
  )
  \bo{c}_\mr{i}
=
  E
  \bo{c}_\mr{i}
&&
  \bo{V}_\mr{ii}
\equiv
  \bo{H}_\mr{ie}
  \bo{R}_\mr{ee}
  \bo{H}_\mr{ei}
\end{align}

\begin{align}
  E
=
  \fr{
    \bo{c}_\mr{i}\dg
    (
      \bo{H}_\mr{ii}
    +
      \bo{V}_\mr{ii}
    )
    \bo{c}_\mr{i}
  }{
    \bo{c}_\mr{i}\cdot
    \bo{c}_\mr{i}
  }
\end{align}


\begin{align*}
  E
=
  \ip{\F|
  (
    1
  +
    \La
  )
  \ol{H}
  |\F}
+
  \ip{\F|\La\ol{H}|\bo{e}}
  \ip{\bo{e}|E - \ol{H}|\bo{e}}^{-1}
  \ip{\bo{e}|\ol{H}|\F}
\end{align*}

\begin{align*}
  \d E
=
  \ip{\F|\La\ol{H}|\bo{e}}
  \ip{\bo{e}|E - \ol{H}|\bo{e}}^{-1}
  \ip{\bo{e}|\ol{H}|\F}
\approx
  \ip{\F|\La\ol{H}\ord{1}|\bo{e}}
  \ip{\bo{e}|E\ord{0} - \ol{H}\ord{0}|\bo{e}}^{-1}
  \ip{\bo{e}|\ol{H}\ord{m}|\F}
\end{align*}


\begin{align}
  \ol{H}
=
  E_\mr{ref}
+
  H_0
+
  (
    H_0
    T
  +
    V_\mr{c}\,
    \mr{exp}(T)
  )_\mr{C}
\end{align}

\begin{align}
  \ol{H}\ord{0}
=
  E_\mr{ref}
+
  H_0
&&
  \ol{H}\ord{1}
=
  (H_0T\ord{1})_\mr{C}
+
  V_\mr{c}
&&
  \ol{H}\ord{2}
=
  (
    H_0
    T\ord{2}
  +
    V_\mr{c}\,
    T\ord{1}
  )_\mr{C}
\end{align}

Structure of matrix
\begin{align*}
  \ol{\bo{H}}_\mr{c}
=
\pma{
  E_\mr{c} & \fwbr{\F}\ol{H}_\mr{c}\fwkt{\bo{i}} & \fwbr{\F}\ol{H}_\mr{c}\fwkt{\bo{e}} \\
  \bo{0} & \fwbr{\bo{i}}\ol{H}_\mr{c}\fwkt{\bo{i}} & \fwbr{\bo{i}}\ol{H}_\mr{c}\fwkt{\bo{e}} \\
  \fwbr{\bo{e}}\ol{H}_\mr{c}\fwkt{\F} & \fwbr{\bo{e}}\ol{H}_\mr{c}\fwkt{\bo{i}} & \fwbr{\bo{e}}\ol{H}_\mr{c}\fwkt{\bo{e}} \\
}
&&
  \bo{i}
\equiv
\pma{
  \bo{s}_1 & \cd & \bo{s}_m
}
&&
  \bo{e}
\equiv
\pma{
  \bo{s}_{m-1} & \cd & \bo{s}_n
}
&&
  \bo{s}_h
\equiv
\pma{
  \cd
\
  \F_{i_1\cd i_h}^{a_1\hspace{-1pt}\cd a_h}
\
  \cd
}
\end{align*}

\newpage
\appendix
\section{Fa\`a di Bruno's formula}

\begin{thm}
\thmtitle{Fa\`a di Bruno's formula}
\begin{align}
  \pd{^n}{x_1\cd \pt x_n}
  f(g(\bm{x}))
=
  \sum_{k=1}^n
  \sum_{(\bm{x}_1,\ld,\bm{x}_k)}^{\mc{P}_k(\bm{x})}
  f\ord{k}(g(\bm{x}))
  \prod_{i=1}^k
  \pd{
    ^{|\bm{x}_i|}
    g(\bm{x})
  }{
    x_{i,1}
  \cd
    \pt
    x_{i,|\bm{x}_i|}
  }
\end{align}
\end{thm}


\end{document}