\documentclass[11pt]{article}
\usepackage[cm]{fullpage}
%%AVC PACKAGES
\usepackage{avcgreek}
\usepackage{avcfonts}
\usepackage{avcmath}
\usepackage[numberby=section,skip=9pt plus 2pt minus 7pt]{avcthm}
\usepackage{qcmacros}
\usepackage{goldstone}
%%MACROS FOR THIS DOCUMENT
\numberwithin{equation}{section}
\usepackage[
  margin=1.5cm,
  includefoot,
  footskip=30pt,
  headsep=0.2cm,headheight=1.3cm
]{geometry}
\usepackage{fancyhdr}
\pagestyle{fancy}
\fancyhf{}
\fancyhead[LE,RO]{Quiz 9, Handout 1: Response theory}
\fancyfoot[CE,CO]{\thepage}
\usepackage{url}
\makeatother
\newcommand{\resolventline}[2][1]{
  \tikz[overlay]{
      \draw[thick,flexdotted] (0,-1ex) to ++(0,#1*4.5ex) node[above,inner sep=1pt] {#2};
  }
}
\usepackage{accents}
\newcommand{\oc}[1]{\ensuremath{\accentset{\circ}{#1}}}
\newcommand{\wtl}[1]{\ensuremath{\widetilde{#1}}}

\begin{document}

\setlength{\abovedisplayskip}{5pt}
\setlength{\belowdisplayskip}{5pt}


\setcounter{section}{8}
\section{Response theory}


\newpage
\appendix

\section{Time-dependent perturbation theory}

\begin{rmk}
In an time-varying field, the electronic wavefunction is no longer simply an eigenfunction of the Hamiltonian.
This more general system is described by the following \textit{time-dependent Schr\"odinger equation}
\begin{align}
\label{eq:schrodinger-equation}
  H(t)
  \Y(t)
=
  i
  \pd{\Y(t)}{t}
&&
  H(t)
=
  H
+
  V(t)
\end{align}
where $H$ is the usual electronic Hamiltonian and $V(t)$ describes the interaction with the external field.
If the electrons are prepared in a particular state $\Y_0$ at some time $t_0$, the system is completely described by a \textit{time-evolution operator}.
\begin{align}
  \Y(t)=U(t,t_0)\Y_0
&&
  \Y_0
=
  \Y(t_0)
\end{align}
The following discussion shows how to expand this operator in orders of the perturbing interaction, $V(t)$.
\end{rmk}

\begin{dfn}
\thmtitle{Interaction picture}
The \textit{interaction picture} results from to the following similarity transformation.
\begin{align}
  \tl{\Th}(t)
\equiv
  e^{+iHt}
  \Th(t)
&&
  \tl{W}(t)
\equiv
  e^{+iHt}
  W(t)
  e^{-iHt}
\end{align}
Expanding the Schr\"odinger equation in the interaction picture yields the the \textit{Schwinger-Tomonaga equation}.
\begin{align}
\label{eq:schwinger-tomonaga}
  \tl{V}(t)
  \tl{\Y}(t)
=
  i
  \pd{\tl{\Y}(t)}{t}
\end{align}
Multiplying both sides by $-i$ and integrating from $t_0$ to $t$ yields a recursive equation for the interaction-picture wavefunction
\begin{align}
\label{eq:integrated-schwinger-tomonaga-equation}
  \tl{\Y}(t)
-
  \tl{\Y}(t_0)
=
-
  i
  \int_{t_0}^t
  dt'\,
  \tl{V}(t')
  \tl{\Y}(t')
\end{align}
from which we infer
$
  \tl{U}(t,t_0)
=
  1
-
  i
  \int_{t_0}^t
  dt'
  \tl{V}(t')\,
  \tl{U}(t',t_0)
$.
Infinite recursion yields the following.
\begin{align}
\label{eq:time-evolution-infinite-recursion}
  \tl{U}(t,t_0)
=
  \sum_{n=0}^\infty
  (-i)^n
  \int_{t_0}^t
  dt_1
  \int_{t_0}^{t_1}
  dt_2
  \cd
  \int_{t_0}^{t_{n-1}}
  dt_n
  \,
  \tl{V}(t_1)
  \cd
  \tl{V}(t_n)
\end{align}
\end{dfn}

\begin{dfn}
\label{dfn:time-ordering}
\thmtitle{Time-ordering}
Let 
$
  \tl{q}_1(t_1)\cd \tl{q}_n(t_n)
$
be a string of particle-hole operators in the interaction picture.\footnotemark
\footnotetext{
As in
$
  \tl{q}(t)
\equiv
  e^{+iHt}
  q
  e^{-iHt}
$
for some
$q\in\{a_p\}\cup\{a_p\dg\}$.
}
The \textit{time-ordering map} takes this string into
$
  \mc{T}\{
  \tl{q}_1(t_1)\cd \tl{q}_n(t_n)
  \}
\equiv
  \e_\pi \,\tl{q}_{\pi(1)}(t_{\pi(1)})\cd \tl{q}_{\pi(n)}(t_{\pi(n)})
$,
where $\pi\in\mr{S}_n$ is a permutation that puts the time arguments in reverse-chronological order, $t_{\pi(1)}>\cd>t_{\pi(n)}$.
\end{dfn}

\begin{ntt}
The following notation proves convenient for manipulating multiple integrals
\begin{align}
  \int_{t_1t_2t_3\ld}^{[t_0,t]}
  dt_1dt_2dt_3\cd
\equiv
  \int_{t_0}^t
  dt_1
  \int_{t_0}^t
  dt_2
  \int_{t_0}^t
  dt_3
  \cd
&&
  \int_{t_1>t_2>t_3>\cd}^{[t_0,t]}
  dt_1dt_2dt_3\cd
\equiv
  \int_{t_0}^t
  dt_1
  \int_{t_0}^{t_1}
  dt_2
  \int_{t_0}^{t_2}
  dt_3
  \cd
\end{align}
which is defined by analogy with summation notation
$
  \sum_{i_1i_2i_3\cd}^{\{n_0,\ld,n\}}
$
and
$
  \sum_{i_1>i_2>i_3\cd}^{\{n_0,\ld,n\}}
$.
That is, the $t_i$'s are dummy variables which we integrate over all values in $[t_0,t]$ satisfying a condition, such as $t_1>t_2>t_3>\cd$.
Then the identity
\begin{align}
\label{eq:integral-identity}
  \int_{t_1\cd t_n}^{[t_0,t]}
  dt_1\cd t_n\,
  f(t_1\cd t_n)
=
  \sum_\pi^{\mr{S}_n}
  \int_{t_{\pi(1)}>\ld>t_{\pi(n)}}^{[t_0,t]}
  dt_1\cd t_n\,
  f(t_1\cd t_n)
\end{align}
follows from considering all possible chronologies for $t_1,\ld,t_n$ in the unrestricted integral.\footnotemark
\footnote{
  The corresponding summation identity would be
$
  \sum_{i_1\neq i_2\neq i_3\neq\cd}^{\{n_0,n\}}
=
  \sum_{\pi}^{\mr{S}_n}
  \sum_{i_{\pi(1)}> i_{\pi(2)}> i_{\pi(3)}>\cd}^{\{n_0,n\}}
$.
  The unrestricted integral is equivalent to an integral over $t_1\neq t_2\neq t_3\neq\cd$ because individual integrand values have ``measure zero'':
$
  \int_{t_j}^{t_j}
  dt_i
=
  0
$.
}
\end{ntt}


\begin{prop}
\thmtitle{The Dyson series}
\thmstatement{
$\ds{
  \tl{U}(t,t_0)
=
  \mc{T}\{
    e^{
    -
      i
      \int_{t_0}^t
      dt'\,
      \tl{V}(t')
    }
  \}
}$
}
\thmproof{
   Expanding the exponential in a Taylor series and applying equation~\ref{eq:integral-identity} gives the following
\begin{align}
  \sum_{n=0}^\infty
  \fr{(-i)^n}{n!}
  \int_{t_1\cd t_n}^{[t_0,t]}
  dt_1\cd dt_n\,
  \mc{T}\{
    \tl{V}(t_1)
    \cd
    \tl{V}(t_n)
  \}
=
  \sum_{n=0}^\infty
  \fr{(-i)^n}{n!}
  \sum_\pi^{\mr{S}_n}
  \int_{t_{\pi(1)}>\ld>t_{\pi(n)}}^{[t_0,t]}
  dt_1\cd dt_n\,
  \mc{T}\{
    \tl{V}(t_1)
    \cd
    \tl{V}(t_n)
  \}
\end{align}
which simplifies to equation~\ref{eq:time-evolution-infinite-recursion} because all $n!$ terms in the sum over $\pi$ are equivalent by \cref{dfn:time-ordering}.\footnote{We are assuming that $\tl{V}(t)$ is particle-number-conserving, or at least contains only even operator products.}
}
\end{prop}



\newpage
\section{Response functions}

\begin{rmk}
A convenient starting point for \textit{response theory} is to cast the interaction Hamiltonian in the following form
\begin{align}
\label{}
  V(t)
=
  \sum_\b
  V_\b
  f_\b(t)
\end{align}
where $\{V_\b\}$ is a set of one-particle operators and the $f_\b(t)$'s are scalar-valued \textit{time envelopes}.
\end{rmk}

\begin{ex}
Interactions with electric and magnetic fields are approximately described by the following Hamiltonians
\begin{align}
\begin{array}{r@{\ }l@{\ }l@{\hspace{2cm}}r@{\ }l@{\hspace{2cm}}r@{\ }l}
  V_{\bo{E}}(t)
=&
-\,
  \bm{\mu}\cdot\bo{E}(t)
&=
-
  \sum_\b
  \mu_\b
  \mc{E}_\b(t)
&
  \bm{\mu}
&=
\ds{
  \sum_{pq}
  \ip{\y_p|\op{\bm{\mu}}\,|\y_q}
  a_p\dg a_q
}
&
  \op{\bm{\mu}}
&=
-
  \op{\bo{r}}
\\
  V_{\bo{B}}(t)
=&
-
  \bm{m}\cdot\bo{B}(t)
&=
-
  \sum_\b
  m_\b
  \mc{B}_\b(t)
&
  \bm{m}
&=
\ds{
  \sum_{pq}
  \ip{\y_p|\op{\bm{m}}|\y_q}
  a_p\dg a_q
}
&
  \op{\bm{m}}
&=
-
  \dfr{\op{\bo{r}}\times\op{\bo{p}}}{2}
\end{array}
\end{align}
where $\bm{\mu}$ and $\bm{m}$ are the electric and magnetic dipole operators and the field components, $\mc{E}_\b(t)$ and $\mc{B}_\b(t)$, are scalar-valued functions of time.
This is known as the \textit{dipole approximation}.
\end{ex}

\begin{rmk}
A convenient set of boundary conditions for response theory turns the interaction off in the infinite past and requires that the system begins in a stationary state, which is usually chosen to be the ground state.
\begin{align}
  \lim_{t\rightarrow-\infty}
  f_\b(t)
=
  0
&&
  \lim_{t\rightarrow-\infty}
  \tl{\Y}(t)
=
  \Y_0
&&
  H
  \Y_k
=
  E_k
  \Y_k
\end{align}
This can be imposed on any time envelope with a finite $t\rightarrow-\infty$ limit by building in a factor $e^{-\ev t}$ where $\ev$ is a real number.
For sufficiently small $\ev$, this new envelope will match the old one to arbitrary precision in an arbitrarily wide window about the time origin.
The Dyson series for the wavefunction can then be written in the form
\begin{align}
  \tl{\Y}(t)
=
  \lim_{t_0\rightarrow-\infty}
  \tl{U}(t,t_0)
  \Y_0
=
  \sum_{n=0}^\infty
  \fr{(-i)^n}{n!}
  \int_{\mb{R}^n}
  dt_1\cd dt_n\,
  \th(t-t_1)
  \cd
  \th(t-t_n)\,
  \mc{T}\{
    \tl{V}(t_1)
    \cd
    \tl{V}(t_n)
  \}
  \Y_0
\end{align}
where
$
  \th(x)
=
  \int_{-\infty}^x
  dx'
  \d(x')
$
is the Heaviside step function, which here enforces an upper limit of $t$ for each integral over $t_i$.
\end{rmk}

\begin{dfn}
Any perturbation-dependent quantity $X(t)$ can be expanded in a Taylor expansion of the time-envelopes
\begin{align}
\label{eq:}
  X(t)
=
  \sum_{n=0}^\infty
  \fr{1}{n!}
  \sum_{\b_1,\ld,\b_n}
  \int_{\mb{R}^n}
  dt_1\cd t_n\,
  f_{\b_1}(t_1)
  \cd
  f_{\b_n}(t_n)\,
  X^{\b_1\cd \b_n}_{t;t_1\cd\,t_n}
&&
  X^{\b_1\cd \b_n}_{t;t_1\cd\,t_n}
\equiv
  \left.
  \fd{^n
    X(t)
  }{
    f_{\b_1}(t_1)
    \cd
    df_{\b_n}(t_n)
  }
  \right|_{\bm{f}=\bo{0}}
\end{align}
where the coefficients $X_{t;t_1\cd t_n}^{\b_1\cd \b_n}$ are $n\eth$-order \textit{responses} or \textit{response functions}, although the latter term is often restricted to the case where $X(t)$ is an observable expectation value, $\ip{W}(t)=\ip{\Y(t)|W(t)|\Y(t)}$, in which case we denote the response with double-brackets
$
  \iip{\tl{W}(t); \tl{V}_{\b_1}(t_1),\ld,\tl{V}_{\b_n}(t_n)}
\equiv
  \ip{W}_{t;t_1\cd t_n}^{\b_1\cd \b_n}
$.
In some contexts these property response functions are referred to as \textit{retarded propagators} or \textit{retarded Green's functions}.
\end{dfn}

\begin{ex}
\begin{align}
  \Y^{\b_1\cd \b_n}_{t; t_1\cd t_n}
=
  (-i)^n
  \th(t-t_1)
  \cd
  \th(t-t_n)\,
  \mc{T}\{
    \tl{V}_{\b_1}(t_1)
  \cd
    \tl{V}_{\b_n}(t_n)
  \}
  \Y_0
\end{align}
\end{ex}

\begin{ntt}
\begin{align}
  \iip{\tl{W}(t); \tl{V}_{\b_1}(t_1),\ld,\tl{V}_{\b_n}(t_n)}
\equiv
  \left.
  \fd{^n
    \ip{\Y(t)|W(t)|\Y(t)}
  }{
    f_{\b_1}(t_1)
    \cd
    df_{\b_n}(t_n)
  }
  \right|_{\bm{f}=\bo{0}}
\end{align}
\end{ntt}

\begin{ex}
\begin{align}
  \tl{\Y}^{\b}_{t;t'}
=
  \left.
  \pd{\tl{\Y}(t)}{f_\b(t')}
  \right|_{\bm{f}=\bo{0}}
=
\end{align}
\end{ex}


\newpage
\section{Fourier transforms}


\begin{rmk}
\begin{align}
  f_\b(t)
=
  \int_{-\infty}^\infty
  d\w\,
  f_\b(\w)
  e^{-i\w t}
&&
  f_\b(\w)
\equiv
  (2\pi)^{-1}
  \int_{-\infty}^\infty
  dt\,
  f_\b(t)
  e^{+i\w t}
&&
  \w
\equiv
  \mr{Re}(\w)
+
  i\ev
\end{align}
$
  f_\b(-\w)
=
  f_\b^*(\w)
$\\
Footnote:
Fourier transforms can always be verified using
$\ds{
  \int_\mb{R}
  dk\,
  e^{ikx}
=
  2\pi\,\d(x)
}$
\end{rmk}


\begin{rmk}
Define $\ta_j\equiv t_j-t$
and note that
$
  X^{\b_1\cd \b_n}_{t;t_1\cd t_n}
=
  X^{\b_1\cd \b_n}_{0;\ta_1\cd \ta_n}
$
\begin{align}
  X^{\b_1\cd \b_n}_{0;\ta_1\cd \ta_n}
\equiv
  (2\pi)^{-n}
  \int_{\mb{R}^n}
  d\w_1\cd d\w_n\,
  X^{\b_1\cd \b_n}_{\w_1\cd\w_n}
  e^{+i\sum_j\w_j\ta_j}
&&
  X^{\b_1\cd \b_n}_{\w_1\cd\w_n}
\equiv
  \int_{\mb{R}^n}
  d\ta_1\cd d\ta_n\,
  X^{\b_1\cd \b_n}_{0;\ta_1\cd \ta_n}
  e^{-i\sum_j\w_j\ta_j}
\end{align}
\end{rmk}

\begin{rmk}
\thmstatement{
$\ds{
  \int_{\mb{R}^n}
  dt_1\cd dt_n\,
  f_{\b_1}(t_1)
  \cd
  f_{\b_n}(t_n)\,
  X_{t;t_1\cd t_n}^{\b_1\cd\b_n}
=
  \int_{\mb{R}^n}
  d\w_1\cd d\w_n\,
  f_{\b_1}(\w_1)
  \cd
  f_{\b_n}(\w_n)
  X^{\b_1\cd \b_n}_{\w_1\cd\w_n}
  e^{-i\sum_j\w_jt}
}$
}
\end{rmk}

\newpage
\section{Complex Calculus}


\end{document}