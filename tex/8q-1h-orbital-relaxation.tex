\documentclass[11pt]{article}
\usepackage[cm]{fullpage}
%%AVC PACKAGES
\usepackage{avcgreek}
\usepackage{avcfonts}
\usepackage{avcmath}
\usepackage[numberby=section,skip=9pt plus 2pt minus 7pt]{avcthm}
\usepackage{qcmacros}
\usepackage{goldstone}
%%MACROS FOR THIS DOCUMENT
\numberwithin{equation}{section}
\usepackage[
  margin=1.5cm,
  includefoot,
  footskip=30pt,
  headsep=0.2cm,headheight=1.3cm
]{geometry}
\usepackage{fancyhdr}
\pagestyle{fancy}
\fancyhf{}
\fancyhead[LE,RO]{Quiz 8, Handout 1: Orbital relaxation}
\fancyfoot[CE,CO]{\thepage}
\usepackage{url}
\makeatother
\newcommand{\resolventline}[2][1]{
  \tikz[overlay]{
      \draw[thick,flexdotted] (0,-1ex) to ++(0,#1*4.5ex) node[above,inner sep=1pt] {#2};
  }
}

\begin{document}

\setlength{\abovedisplayskip}{5pt}
\setlength{\belowdisplayskip}{5pt}


\setcounter{section}{7}
\section{Orbital relaxation}


\begin{rmk}
\thmtitle{Orbital relaxation}
According to the Thouless theorem (\cref{appendix:thouless}), the effect of the singles CC operator is to transform the orbitals of the reference determinant into a new set $\{\widetilde{\y}_i\}$ by ``mixing in'' some of the virtual orbitals.
\begin{align}
  \Y_\mr{CC}
=
  \mr{exp}(T_2 + T_3 + \cd)
  \widetilde{\F}
&&
  \widetilde{\F}
\equiv
  \mr{exp}(T_1)
  \F
=
  \tfr{1}{\sqrt{n!}}\,
  \mr{det}(\widetilde{\y}_1\cd \widetilde{\y}_n)
&&
  \widetilde{\y}_i
=
  \y_i
+
  \sum_a
  \y_a
  t_a^i
\end{align}
This can be thought of as ``relaxing'' the orbitals in the presence of electron correlation.
The size of this \textit{orbital relaxation effect} can be monitored as the root mean square difference from the reference orbitals, which is known as the \textit{$\mc{T}_1$ diagnostic}.
\begin{align}
  \mc{T}_1
\equiv
  \sqrt{
  \fr{1}{n}
  \sum_{i=1}^n
  \|\widetilde{\y}_i - \y_i\|^2
  }
=
  \fr{\|\bo{t}_1\|}{\sqrt{n}}
\end{align}
Significant orbital relaxation generally indicates that the reference determinant forms a poor approximation to the wavefunction, which can lead to large errors for low-order truncated methods like CCSD or CCSD(T).
In closed-shell systems, significant orbital relaxation is usually associated with an inherent \textit{multireference character}, which means that no single determinant dominates the wavefunction with any choice of orbitals.
Empirically, $\mc{T}_1\geq 0.02$ is considered large in this context.
In open-shell systems, mean-field methods like Hartree-Fock theory are often deficient even for non-multireference systems.
In this case, orbital relaxation effects can generally be cured by choosing a new determinant which is optimized in the presence of dynamical\footnote{As opposed to mean-field.} electron correlation.
\end{rmk}

\begin{rmk}
\thmtitle{Brueckner and orbital-optimized methods}
The two most common ways of defining an ideal reference determinant for the correlated wavefunction are the \textit{best overlap criterion} and the \textit{best energy criterion}.\footnote{See \url{https://en.wikipedia.org/wiki/Arg_max} for the notation used here.}
\begin{align}
\label{eq:brueckner-and-oo-general-condition}
  \F_{\mr{B}}
=
  \underset{\F}{\arg\max}\,
  \ip{\F|\Y}
&&
  \F_{\mr{O}}
=
  \underset{\F}{\arg\min}\,
  \ip{\Y|H|\Y},\ \
  \Y
=
  \W\,\F
\end{align}
The \textit{best overlap} or \textit{Brueckner determinant}, $\F_\mr{B}$, has maximum overlap with the wavefunction.
The \textit{best energy} or \textit{orbital-optimized determinant}, $\F_\mr{O}$, yields the lowest energy for a given Ansatz.
The spin-orbital sets from which these determinants are constructed will be denoted $\{\y_p\}_\mr{B}$ and $\{\y_p\}_\mr{O}$.
\end{rmk}

\begin{rmk}
\Cref{appendix:orbital-rotations}
shows that non-redundant transformations of the orbitals in $\F$ can be parametrized as
\begin{align}
  \F(\bo{x})
=
  \mr{exp}(X - X\dg)
  \F
\end{align}
where $X$ has the form of a $T_1$ operator.
This parametrization can be substituted into equation~\ref{eq:brueckner-and-oo-general-condition} to derive the explicit conditions satisfied by Brueckner orbitals and energy-optimized orbitals.\footnote{
  $\kt{\Y(\bm{x})}=\mr{exp}(X-X\dg)\kt{\Y}$ applies the orbital rotation to all of the determinants in the wavefunction expansion.
  }
\begin{align*}
  \{\y_p\}_\mr{B}:\,\,
  \left.
  \pd{}{x_a^{i*}}
  \ip{\F(\bo{x})|\Y}
  \right|_{\bo{x}=\bo{0}}
=
  \ip{\F_i^a|\Y}
\overset{!}{=}
  0
&&
  \{\y_p\}_\mr{O}:\,\,
  \left.
  \pd{}{x_a^{i*}}
  \ip{\Y(\bm{x})|H|\Y(\bm{x})}
  \right|_{\bo{x}=\bo{0}}
=
  \ip{\Y|\,[a^i_a, H]\,|\Y}
\overset{!}{=}
  0
\end{align*}
\end{rmk}





\newpage
\appendix

\section{The Thouless theorem}
\label{appendix:thouless}

\begin{ntt}
\label{ntt:orbital-transformation}
Let $\bm{\y}$ be a row vector of orthonormal spin-orbitals, $(\bm{\y})_p=\y_p$, which can be split into occupied and virtual blocks as $\bm{\y}=[\bm{\y}_\mr{o}\ \bm{\y}_\mr{v}]$.
Other sets of spin-orbitals are related to this one by a transformation
$
  \kt{\bm{\y}'}
=
  \kt{\bm{\y}}\,
  \bo{U}
$
which is unitary if the primed orbitals are also orthonormal.
Let $\F'$ be the \textit{transformed reference determinant}, constructed from the first $n$ orbitals in the transformed space.
Then the occupied and virtual orbitals of the transformed space are given by
\begin{align}
\label{eq:block-transformation}
  \kt{\bm{\y}'_\mr{o}}
=
  \kt{\bm{\y}_\mr{o}}\,
  \bo{U}_\mr{oo}
+
  \kt{\bm{\y}_\mr{v}}\,
  \bo{U}_\mr{vo}
&&
  \kt{\bm{\y}'_\mr{v}}
=
  \kt{\bm{\y}_\mr{o}}\,
  \bo{U}_\mr{ov}
+
  \kt{\bm{\y}_\mr{v}}\,
  \bo{U}_\mr{vv}
\end{align}
in terms of the occupied and virtual blocks of the transformation.
This kind of unitary transformation of the spin-orbital basis is sometimes referred to as an \textit{orbital rotation}.

%If $\bm{\y}$ is constructed from a basis set $\bm{\x}=[\x_\mu]$ as $\bm{\y}=\bm{\x}\,\bo{C}$ where $\bo{C}$, then the coefficients of $\bm{\y}'$ are given by $\bo{C}'=\bo{C}\bo{U}$.

\end{ntt}


\begin{thm}
\label{thm:thouless}
\thmtitle{The Thouless theorem}
\begin{enumerate}
\item
\label{item:thouless-part-1}
\thmstatement{
  The function
  $e^{T_1}\F$
  is an intermediately normalized determinant
  $
    \tfr{1}{\sqrt{n!}}
    \mr{det}(\widetilde{\y}_1\cd\widetilde{\y}_n)
  $
  with orbitals
  $
    \widetilde{\y}_i
  =
    \y_i
  +
    \sum_a
    \y_a
    t_a^i
  $.
}
\thmproof{
  Intermediate normalization follows from
  $
    \ip{\F|e^{T_1}\F}
  =
    1
  $.
  This function has the form of a determinant
\begin{align*}
  e^{T_1}
  \kt{\F}
=
  e^{\sum_{a}t_a^1 a_1^a + \cd + \sum_a t_a^n a_n^a}
  a_1\dg
  \cd
  a_n\dg
  \kt{\vac}
=
  \widetilde{a}_1\dg
  \cd
  \widetilde{a}_n\dg
  \kt{\vac}
=
  \kt{\widetilde{\F}}
&&
  \widetilde{a}_i\dg
\equiv
  \mr{exp}(\ts{\sum_{a}t_a^i a_i^a})\,
  a_i\dg
\end{align*}
  since $\sum_a t_a^ia_i^a$ commutes with all creation operators except $a_i\dg$.
  The transformed orbitals are given by
\begin{align*}
  \kt{\widetilde{\y}_i}
=
  \widetilde{a}_i\dg
  \kt{\vac}
=
  \mr{exp}(\ts{\sum_{a}t_a^i a_a\dg a_i})\,
  a_i\dg
  \kt{\vac}
=
  (\ts{
    1
  +
    \sum_a
    t_a^i
    a_a\dg
    a_i
  })\,
    a_i\dg
  \kt{\vac}
=
  \kt{\y_i}
+
  \ts{\sum_a}
  t_a^i
  \kt{\y_a}
\end{align*}
 using $a_i^2=0$ and $a_ia_i\dg\kt{\vac}=\kt{\vac}$.
}

\item
\thmstatement{
  Any intermediately normalized determinant
  $
    \widetilde{\F}
  =
    \tfr{1}{\sqrt{n!}}
    \mr{det}(\widetilde{\y}_1\cd\widetilde{\y}_n)
  $
  can be written as $e^{T_1}\,\F$.
}
\thmproof{
  Intermediate normalization is only possible if $\widetilde{\F}$ has non-zero overlap with the reference determinant.
  Therefore, $\widetilde{\F}$ can be written as
  $\F'/\ip{\F|\F'}$
  where $\F'$ is a Slater determinant.
  The normalization factor is given by
\begin{align*}
\ts{
  \ip{\F|\F'}
=
  \tfr{1}{n!}
  \sum_{\pi,\si}^{\mr{S}_n}
  \e_\pi
  \e_\si
  \ip{\y_{\pi(1)}|\y'_{\si(1)}}
  \cd
  \ip{\y_{\pi(n)}|\y'_{\si(n)}}
=
  \sum_{\si}^{\mr{S}_n}
  \e_\si
  \ip{\y_{1}|\y'_{\si(1)}}
  \cd
  \ip{\y_{n}|\y'_{\si(n)}}
=
  \mr{det}(\bo{U}_{\mr{oo}})
}
\end{align*}
  following \cref{ntt:orbital-transformation}.
  Therefore,
  $
    \widetilde{\F}
  =
    \F'/
    \mr{det}(\bo{U}_\mr{oo})
  =
    \F'\,
    \mr{det}(\bo{U}_\mr{oo}^{-1})
  $
  and the rows of $\widetilde{\F}$ are given by
\begin{align*}
  \kt{\bm{\widetilde{\y}}_\mr{o}}
=
  \kt{\bm{\y}'_\mr{o}}\,
  \bo{U}_\mr{oo}^{-1}
=
  \kt{\bm{\y}_\mr{o}}\,
+
  \kt{\bm{\y}_\mr{v}}\,
  \bo{U}_\mr{vo}
  \bo{U}_\mr{oo}^{-1}
\end{align*}
  where we have expanded $\kt{\bm{\y}_\mr{o}'}$ according to eq~\ref{eq:block-transformation}.
  The columns of this equation are
  $
    \widetilde{\y}_i
  =
    \y_i
  +
    \sum_a
    \y_a\,
    (\bo{U}_\mr{vo}\bo{U}_\mr{oo}^{-1})_{ai}
  $.
  Referring back to the first proposition, this shows that
  $
    \widetilde{\F}
  =
    e^{T_1}\F
  $
  with
  $
    t_a^i
  =
    (\bo{U}_\mr{vo}\bo{U}_\mr{oo}^{-1})_{ai}
  $.
}
\end{enumerate}
\end{thm}




\newpage
\section{Orbital rotations}
\label{appendix:orbital-rotations}


\begin{dfn}
\label{dfn:normal-matrix}
\thmtitle{Normal matrix}
A square matrix satisfying $\bo{N}\dg\bo{N}=\bo{N}\bo{N}\dg$ is termed \textit{normal}.
Several important kinds of matrices meet this criterion:
\textit{Hermitian matrices}, $\bo{H}\dg=\bo{H}$;
\textit{anti-Hermitian matrices}, $\bo{A}\dg=-\bo{A}$;
and
\textit{unitary matrices}, $\bo{U}\dg=\bo{U}^{-1}$.
Note that Hermitian and anti-Hermitian matrices can always be written as $\bo{X}+\bo{X}\dg$ and $\bo{X}-\bo{X}\dg$.
\end{dfn}

\begin{rmk}
\label{rmk:spectral-theorem}
The Spectral Theorem\footnote{See \url{https://en.wikipedia.org/wiki/Spectral_theorem}} says that every normal matrix can be diagonalized by a unitary transformation.
A direct corollary\footnote{Since there exists a basis in which $\bo{N}$ is diagonal, statements about $\bo{N}$ translate into statements about its eigenvalues.} is that the eigenvalues of Hermitian, anti-Hermitian, and unitary matrices can be written as follows.
\begin{align}
  h^*
=
  h
\implies
  h
=
  \f
&&
  a^*
=
-
  a
\implies
  a
=
  i\f
&&
  u^*
=
  u^{-1}
\implies
  u
=
  e^{i\f}
&&
  \f
\in
  \mb{R}
\end{align}
In words, Hermitian eigenvalues are real, anti-Hermitian eigenvalues are pure imaginary, and unitary eigenvalues lie on the unit circle.
Note that unitary eigenvalues have the form $u=\mr{exp}(a)$ where $a$ is an anti-Hermitian eigenvalue.
This implies that any unitary matrix $\bo{U}$ can be written as $\mr{exp}(\bo{A})$, where $\bo{A}$ is anti-Hermitian.
\end{rmk}

\begin{rmk}
According to \cref{dfn:normal-matrix} and \cref{rmk:spectral-theorem}, unitary transformations of the spin-orbitals can be parametrized as
\begin{align}
\label{eq:spin-orbital-transformation}
  \y_p'
=
  \sum_q
  \y_q
  (\mr{exp}(\bo{X} - \bo{X}\dg))_{qp}
\end{align}
in terms a square matrix $\bo{X}$.
The anti-Hermitian form of this parametrization leads to redundancies.
In particular, notice that $\bo{X}=[z\,\d_{pq}]$ generates the same transformation as $\bo{X}\dg=[-z^*\,\d_{qp}]$.
These redundancies are eliminated by setting the upper or lower triangle of $\bo{X}$ to zero.
For most single-reference methods, transformations within the occupied and virtual blocks are also redundant because they don't change the energy.\footnote{This is what allows us to diagonalize the Fock matrix in canonical Hartree-Fock theory.}
Setting all but one of the off-diagonal blocks to zero yields a non-redundant parametrization.
Here, we will choose $\bo{X}_\mr{vo}=[x_a^i]$.
\end{rmk}


\begin{prop}
\label{prop:creation-operator-similarity-transform}
\thmstatement{
The identity\ \
$\ds{
  \mr{exp}(G)\,a_p\dg\,\mr{exp}(-G)
=
  \sum_q
  a_q\dg\,
  (\mr{exp}(\bo{G}))_{qp}
}$\
holds for any
$
  G
=
  \sum_{pq}
  (\bo{G})_{pq}\,
  a_p\dg a_q
$.
}\vspace{3pt}
\thmproof{
  This follows from
  $
    [G,\cdot\,]^m(a_p\dg)
  =
    \sum_q
    a_q\dg
    (\bo{G}^m)_{qp}
  $,
  which we will prove by induction.
  For $m=0$ the statement is trivially true.
  If we assume it holds for $m$, then the following shows that it also holds for $m+1$,\footnote{
  The second equality here follows from expanding $G$ and using
$
  [a_r\dg a_s, a_q\dg]
=
  \no{
    a_r\dg
    \ctr{}{a}{_s}{}
    a_s a_q\dg
  }
=
  a_r\dg\,
  \d_{sq}
$.
}
\begin{align*}
\ts{
  [G,\cdot\,]^{m+1}(a_p\dg)
=
  \sum_q
  [G, a_q\dg]\,
  (\bo{G}^m)_{qp}
=
  \sum_{qr}
  a_r\dg
  (\bo{G})_{rq}
  (\bo{G}^m)_{qp}
=
  \sum_r
  a_r\dg
  (\bo{G}^{m+1})_{rp}
}
\end{align*}
  which completes the induction.
  Substituting this result into the Hausdorff expansion of
$
  \mr{exp}(G)\,a_p\dg\,\mr{exp}(-G)
$
  recognizing the Taylor expansion of $\mr{exp}(\bo{G})$ completes the proof.
}
\end{prop}


\begin{rmk}
The identity $a_p\dg\kt{\vac}=\kt{\y_p}$ implies that creation operators transform like orbitals.
Therefore, the creation operator corresponding to $\y'_p$ in equation~\ref{eq:spin-orbital-transformation} is given by
$
  a_p^{\prime\,\dagger}
=
  \sum_q
  a_q\dg
  (\mr{exp}(\bo{X} - \bo{X}\dg))_{qp}
$.
By \cref{prop:creation-operator-similarity-transform}, this is equivalent to
\begin{align}
  a_p^{\prime\,\dagger}
=
  \mr{exp}(X - X\dg)
  a_p\dg\,
  \mr{exp}(X\dg - X)
&&
  X
\equiv
  \sum_{ai}
  x_a^i\,
  a^a_i
\end{align}
where we have eliminated the redundant parameters.
Substituting this into
$
  \kt{\F'_{(p_1\cd p_n)}}
=
  a_{p_1}^{\prime\,\dagger}
  \cd
  a_{p_n}^{\prime\,\dagger}
  \kt{\vac}
$
gives a convenient expression for determinants of the transformed orbitals.\,\footnote{This follows from
$
  \mr{exp}(X\dg - X)
  \mr{exp}(X - X\dg)
=
  1
$
and
$
  \mr{exp}(X\dg - X)
  \kt{\vac}
=
  0
$.
}
\begin{align}
  \kt{\F'_{(p_1\cd p_n)}}
=
  \mr{exp}(X - X\dg)
  \kt{\F_{(p_1\cd p_n)}}
&&
  \br{\F'_{(p_1\cd p_n)}}
=
  \br{\F_{(p_1\cd p_n)}}
  \mr{exp}(X\dg - X)
\end{align}

\end{rmk}





\end{document}
