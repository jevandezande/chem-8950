\documentclass[11pt]{article}
\usepackage[cm]{fullpage}
%%AVC PACKAGES
\usepackage{avcgreek}
\usepackage{avcfonts}
\usepackage{avcmath}
\usepackage[numberby=section]{avcthm}
\usepackage{qcmacros}
\usepackage{goldstone}
%%MACROS FOR THIS DOCUMENT
\numberwithin{equation}{section}
\usepackage[
  margin=1.5cm,
  includefoot,
  footskip=30pt,
  headsep=0.2cm,headheight=1.3cm
]{geometry}
\usepackage{fancyhdr}
\pagestyle{fancy}
\fancyhf{}
\fancyhead[LE,RO]{\textbf{Quiz 4, Part 1}}
\fancyfoot[CE,CO]{\thepage}
\usepackage{url}

\begin{document}

\begin{enumerate}
\item
  Give an example of each of the following.
  \begin{enumerate}
  \item
    A closed, connected graph of at least two operators.
    \vspace{5cm}
  \item
    A Hugenholtz path of at least three lines that doesn't qualify as a Goldstone path.
    \vspace{5cm}
  \item
    Non-equivalent, interchangeable subgraphs, where at least one subgraph contains multiple operators.
    \vspace{5cm}
  \item
    A graph that is disconnected and linked.
    \vspace{5cm}
  \end{enumerate}

\newpage
\item
  Interpret the following graph algebraically, and then simplify your expression as much as possible.\footnotemark
\begin{align*}
\diagram{
  \draw[overhang] (0,-0.5) node[ddot] (c1) {};
  \draw[overhang] (1,-0.5) node[ddot] (c2) {};
  \draw[sawtooth] (0.5,0) node[ddot] (g1) {}
    to (2.0,0) node[ddot] (g2) {};
  \interaction{2}{2c}{(2,-0.5)}{ddot}{overhang};
  \draw[->-=0.7] (c1) to ++(-0.25,1);
  \draw[-<-] (c1) to (g1);
  \draw[->-] (c2) to (g1);
  \draw[-<-=0.7] (c2) to ++(+0.25,1);
  \draw[->-,bend left=40] (2c1) to (g2);
  \draw[-<-,bend right=40] (2c1) to (g2);
  \draw[->-] (2c2) to ++(-0.25,1);
  \draw[-<-] (2c2) to ++(+0.25,1);
}
\end{align*}

\footnotetext{
  The operators in this graph are defined as follows.
  \begin{align*}
  \diagram{
    \interaction{2}{g}{(0,0)}{ddot=white}{sawtooth};
    \draw[->-] (g1) to ++(0,+0.5);
    \draw[-<-] (g1) to ++(0,-0.5);
    \draw[->-] (g2) to ++(0,+0.5);
    \draw[-<-] (g2) to ++(0,-0.5);
  }
  \equiv
    \pr{
      \tfr{1}{2!}
    }^2
    \sum_{pqrs}
    \ol{g}_{pq}^{rs}
    \tl{a}^{pq}_{rs}
  &&
  \diagram{
    \draw[overhang] (0,-0.25) node[ddot] (t1) {};
    \draw[->-] (t1) to ++(-0.25,+0.5);
    \draw[-<-] (t1) to ++(+0.25,+0.5);
  }
  \equiv
    \sum_{ia}
    c_a^i
    \tl{a}^a_i
  &&
  \diagram{
    \interaction{2}{t}{(0,-0.25)}{ddot}{overhang};
    \draw[->-] (t1) to ++(-0.25,+0.5);
    \draw[-<-] (t1) to ++(+0.25,+0.5);
    \draw[->-] (t2) to ++(-0.25,+0.5);
    \draw[-<-] (t2) to ++(+0.25,+0.5);
  }
  \equiv
    \pr{
      \tfr{1}{2!}
    }^2
    \sum_{ijab}
    c_{ab}^{ij}
    \tl{a}_{ij}^{ab}
  \end{align*}
}

\newpage
\item
  Write the following algebraic expression as a graph.\footnotemark
\begin{align*}
  \sum_{\substack{abcd\\ijkl}}
  \ol{v}_{ij}^{ab}
  \ol{w}_{bcd}^{jkl}
  \gno{
    a^{ij^\hole}_{ab^\ptcl}
    a^{b^\ptcl cd}_{j^\hole kl}
  }
\end{align*}

\footnotetext{
  Use the following to denote the operators in your graph.
\begin{align*}
  \pr{\tfr{1}{2!}}^2
  \sum_{pqrs}
  \ol{v}_{pq}^{rs}
  \tl{a}^{pq}_{rs}
\equiv
\diagram{
  \node[draw] (label) at (-0.7,0) {\bm{v}};
  \interaction{2}{v}{(0,0)}{ddot=white}{solid};
  \draw (label) to (v1);
  \draw[->-] (v1) to ++(0,+0.45);
  \draw[-<-] (v1) to ++(0,-0.45);
  \draw[->-] (v2) to ++(0,+0.45);
  \draw[-<-] (v2) to ++(0,-0.45);
}
&&
  \pr{\tfr{1}{3!}}^2
  \sum_{\substack{pqr\\stu}}
  \ol{w}_{pqr}^{stu}
  \tl{a}^{pqr}_{stu}
\equiv
\diagram{
  \node[draw] (label) at (-0.7,0) {\bm{w}};
  \interaction{3}{w}{(0,0)}{ddot=white}{solid};
  \draw (label) to (w1);
  \draw[->-] (w1) to ++(0,+0.45);
  \draw[-<-] (w1) to ++(0,-0.45);
  \draw[->-] (w2) to ++(0,+0.45);
  \draw[-<-] (w2) to ++(0,-0.45);
  \draw[->-] (w3) to ++(0,+0.45);
  \draw[-<-] (w3) to ++(0,-0.45);
}
\end{align*}
}


\end{enumerate}

\end{document}
